%%%%%%%%%%%%%%%%%%%%%%%%%%%%%%%%%%%%%%%%%
% Beamer Presentation
% LaTeX Template
% Version 1.0 (10/11/12)
%
% This template has been downloaded from:
% http://www.LaTeXTemplates.com
%
% License:
% CC BY-NC-SA 3.0 (http://creativecommons.org/licenses/by-nc-sa/3.0/)
%
%%%%%%%%%%%%%%%%%%%%%%%%%%%%%%%%%%%%%%%%%

\documentclass{beamer}

\mode<presentation> {
    
    \usetheme{Berlin} %

    \setbeamertemplate{navigation symbols}{} % To remove the navigation symbols from the bottom of all slides uncomment this line
}

\usepackage{graphicx} % Allows including images
\usepackage{booktabs} % Allows the use of \toprule, \midrule and \bottomrule in tables
\usepackage[czech]{babel}
\usepackage[utf8]{inputenc}
\usepackage[T1]{fontenc}

\newcommand{\mypause}{\pause}
%----------------------------------------------------------------------------------------
%	TITLE PAGE
%----------------------------------------------------------------------------------------

\title[FIplomová práce]{\textbf{Mačkostroj}\\Obhajoba FIplomové práce} 
\author{Poznej FI} % Your name
\institute[FI MU] % Your institution as it will appear on the bottom of every slide, may be shorthand to save space
{
    Fakulta informatiky, Masarykova univerzita \\ % Your institution for the title page
    \medskip
    \textit{poznej@fi.muni.cz} % Your email address
}
\date{} % Date, can be changed to a custom date

\begin{document}
    
    \begin{frame}
    \titlepage % Print the title page as the first slide
\end{frame}

\begin{frame}
\frametitle{Obsah} % Table of contents slide, comment this block out to remove it
\tableofcontents % Throughout your presentation, if you choose to use \section{} and \subsection{} commands, these will automatically be printed on this slide as an overview of your presentation
\end{frame}

%----------------------------------------------------------------------------------------
%	PRESENTATION SLIDES
%----------------------------------------------------------------------------------------

\section{Úvod}
\begin{frame}
    \frametitle{Principy fungování Mačkostroje}
    \begin{block}{Zákon zachovania mačky}
        Mačka vždy dopadne na všechny čtyři
    \end{block}
    \mypause
    \begin{block}{23. Murphyho zákon}
        Chleba vždy spadne na zem namazanou stranou
    \end{block}
    \mypause
    \begin{block}{Oklamova břitva}
        Nejjednodušší řešení je vždy to správné
    \end{block}
\end{frame}

\section{Konstrukce}
\begin{frame}
    \frametitle{Konstrukce prototypu}
    \begin{block}{Kaudální mačkostroj}
        Pro převod energie z rotoru do vnějšího agregátu je realizován pomocí mechanického spojení kaudusu mačky se hřídelí.
    \end{block}
    \mypause
    Obrazek
\end{frame}

\begin{frame}
    \frametitle{Konstrukční omezení rotoru}
    Pro výběr mačky pro kaudální mačkostroj je ovlivněn hlavně charakteristikami jejího ocasu
    \begin{itemize}
        \item \textbf{Délka}  --  pokud má mačka příliš krátký ocas, zůstane viset na hřídeli a nebude pokračovat v pádu
        \item \textbf{Citlivost}  -- veškerá energie generovaná tímto zařízením je přenášena přes ocas. Mačky nemají rády, když je někdo tahá za ocas
    \end{itemize}
\end{frame}

\begin{frame}
    \frametitle{Vnější nádoba mačkostroje}
    V principu není pro kaudální mačkostroj potřeba žádná vnější nádoba:
    \begin{itemize}
        \item Rotor je pevně spojen se hřídelí, takže mačka nemůže uniknout
        \item Nádoba neyahrnuje žádné funkční komponenty
        \mypause
        \item Doporučujeme rotor umístit alespoň do klece, aby nedošlo k roztrhání závěsů rotující mačkou
        \mypause
        \item Pokud se v blízkosti mačkostroje budou pohybovat malé děti nebo Maara, doporučujeme plně obalující nádobu, aby nestrkal dovnitř prsty
    \end{itemize}
\end{frame}

\section{Provoz mačkostroje}
\begin{frame}
    \frametitle{Monitoring a regulace}
    \begin{itemize}
        \item Díky absenci nádoby lze stav mačkostroje monitorovat pomocí systému \textquotedblleft kouknu se a vidím\textquotedblright
        \mypause
        \item Výkon lze jednoduše regulovat přibržďovním rotace hřídele
        \mypause
        \item Díky pevnému spojení hřídele s ocasem lze stroj rychle nouzově zastavit zataháním za hřídel. Mačka se lekne a tím dojde k náhlému přeuspořádání tlapek v rotoru
    \end{itemize}
\end{frame}

\begin{frame}
    \frametitle{Údržba zařízení}
    Tento typ mačkostroje lze velice jednoduše udr6ovat i za běhu
    \begin{itemize}
        \item \textbf{Krmení mačky}  lze provádět za chodu stroje opatrným přiléváním mléka do úst rotoru
        \mypause
        \item \textbf{Doplnění másla}  na chleba zvládne zkušený operátor i za běhu stroje, protože agregát nemá vnější nádobu
        \mypause
        \item \textbf{Výměnu chleba}  není potřeba dělat příliš často, protoče drolící se kousky starého chleba nezanášejí žádné důležité komponenty
    \end{itemize}
\end{frame}

\section{Závěr}
\begin{frame}
    \frametitle{Shrnutí}
    \mypause
    Kaudální varianta mačkostroje je vhodné zařízení do drsnějších podmínek
    \\ \vspace*{5pt} \mypause 
    Málo komponent znamená nízkou náchylnost k poruchám
    \\ \vspace*{5pt} \mypause 
    Lze použít i mačku bey srsti, takže jí není třeba pořád čistit kožich
    \\ \vspace*{5pt} \mypause 
    Má jednoduchou údržbu, kterou není třeba provádět příliš často
\end{frame}

\end{document}