%%%%%%%%%%%%%%%%%%%%%%%%%%%%%%%%%%%%%%%%%
% Beamer Presentation
% LaTeX Template
% Version 1.0 (10/11/12)
%
% This template has been downloaded from:
% http://www.LaTeXTemplates.com
%
% License:
% CC BY-NC-SA 3.0 (http://creativecommons.org/licenses/by-nc-sa/3.0/)
%
%%%%%%%%%%%%%%%%%%%%%%%%%%%%%%%%%%%%%%%%%

\documentclass{beamer}

\mode<presentation> {
    
    \usetheme{Berlin} %
    
    \setbeamertemplate{navigation symbols}{} % To remove the navigation symbols from the bottom of all slides uncomment this line
}

\usepackage{graphicx} % Allows including images
\usepackage{booktabs} % Allows the use of \toprule, \midrule and \bottomrule in tables
\usepackage[czech]{babel}
\usepackage[utf8]{inputenc}
\usepackage[T1]{fontenc}

\newcommand{\mypause}{}
%----------------------------------------------------------------------------------------
%	TITLE PAGE
%----------------------------------------------------------------------------------------

\title[FIplomová práce]{\textbf{Mačkostroj}\\Obhajoba FIplomové práce} 
\author{Poznej FI} % Your name
\institute[FI MU] % Your institution as it will appear on the bottom of every slide, may be shorthand to save space
{
    Fakulta informatiky, Masarykova univerzita \\ % Your institution for the title page
    \medskip
    \textit{poznej@fi.muni.cz} % Your email address
}
\date{} % Date, can be changed to a custom date

\begin{document}
    
    \begin{frame}
    \titlepage % Print the title page as the first slide
\end{frame}

\begin{frame}
\frametitle{Obsah} % Table of contents slide, comment this block out to remove it
\tableofcontents % Throughout your presentation, if you choose to use \section{} and \subsection{} commands, these will automatically be printed on this slide as an overview of your presentation
\end{frame}

%----------------------------------------------------------------------------------------
%	PRESENTATION SLIDES
%----------------------------------------------------------------------------------------

\section{Úvod}
\begin{frame}
\frametitle{Principy fungování Mačkostroje}
\begin{block}{Zákon zachovania mačky}
Mačka vždy dopadne na všechny čtyři
\end{block}
\mypause
\begin{block}{23. Murphyho zákon}
Chleba vždy spadne na zem namazanou stranou
\end{block}
\mypause
\begin{block}{Oklamova břitva}
Nejjednodušší řešení je vždy to správné
\end{block}
\end{frame}

\section{Prototypy}
\begin{frame}
\frametitle{Elektro-magnetický mačkostroj}
    \textbf{Koncept:}  Kromě chleba připevníme na mačku i sadu magnetů. Jejich rotací vzniká elektromagnetické pole, které je na vnějším vinutí převáděno na elektrický proud
    \mypause
    \\
    \vspace*{5pt}
    \textbf{Realizace:}  Připevněním cizího předmětu na rotor se celý koncept rozpadne, protože rotor vždy dopadne na některý z magnetů
    
\end{frame}

\begin{frame}
\frametitle{Extrakce elektrostatické energie}
    \textbf{Koncept:}  Při vhodné volbě vnitřního povrchu nádoby mačkostroje vzniká v kožichu statická elektina -- podobně jako při tření liščího ohonu o ebonitovou tyč
    \mypause
    \\
    \vspace*{5pt}
    \textbf{Realizace:}  Úspěšně jsme zrealizovali velmi krátký běh stroje. Toto zařízení je ale velmi nestabilní, takže uvedení stroje do praxe bude vyžadovat velmi důkladné prozkoumání všech doprovodných jevů a rizik. 
\end{frame}

%--------------------------
\begin{frame}
    \frametitle{Konstrukce}
    \begin{itemize}
        \item \textbf{Vnější nádoba}  byla vyrobena ze 100\% ebonitové síťoviny
        \mypause
        \item \textbf{Chleba}  byl použit typ Šumava, dostupný v libovolném supermarketu
        \mypause
        \item \textbf{Máslo}  je drahé, takže jsme to namazali Ramou
        \mypause
        \item \textbf{Mačku}  jsme si půjčili od sousedů. Kiki je taková velká, rezavá potvora co hrozně škrábe. 
        \mypause
        \item \textbf{Ocas}  rotoru jsme nabarvili na bílo, aby ebonit lépe fungoval
    \end{itemize}
\end{frame}

%------------------------
\section{Poznatky}
\begin{frame}
    \frametitle{Ovládání a nedostatky}
    K ovládání mačkostroje jsme použili kus tuňáka. Vhodným umístěním v agregátu jsme tak mohli ovlivnit chování mačky uvnitř
    \mypause \\ \vspace*{5pt}
    
    Výkon stroje je velmi výrazně ovlivněn náladou mačky uvnitř. Kožich naježené mačky má diametrálně rozdílné chování oproti mačce v klidu. Protože náladu ani názory maček nelze nijak ovlivnit, nemůžeme ani ovládat agregát
    \mypause \\ \vspace*{5pt}
    
    Kvůli nedostatku ebonitových tyčí jsme postavili příliš malý stator. Měli jsme tak problém mačku dostatečně zmačkat, aby se do něj vlezla
    \mypause \\ \vspace*{5pt}
    
    Mačky škrábou. Hlavě když na ně patláte lepidlo
\end{frame}

\begin{frame}
\frametitle{Rizika spojená s používáním mačkostroje I}
    Mačka nemůžeme omezit v pohybu, jinak by princip mačkostroje nefungoval
    \mypause \\ \vspace*{5pt}

    Pokud uteče ze spuštěného mačkostroje, tak začne i s chlebem na hřbetě volně levitovat po místnosti
    \mypause \\ \vspace*{5pt}

    Její naježený kožich je v tento moment plně nabitý elektrostatickou elektřinou 
    \mypause \\ \vspace*{5pt}
\end{frame}

\begin{frame}
\frametitle{Rizika spojená s používáním mačkostroje II}
    Volně levitující naježená mačka nabitá elektrostatickou elektřinou způsobuje rušení signálu WiFi
    \mypause \\ \vspace*{5pt}
    
    Po tomto zjištění jsme neprováděli žádné další experimenty
    \mypause \\ \vspace*{5pt}
    
    Dalším rizikem je, že nepředvidatelnost mačky a složitá regulovatelnost mačkostroje
    \mypause \\ \vspace*{5pt}
    
    Přetížení mačkostroje pak může způsobit lokální zmačkání vesmíru
    \mypause \\ \vspace*{5pt}

    Při použití černé mačky může zmačkání vesmíru vytvořit černou díru
\end{frame}

\section{Závěr}
\begin{frame}
    \frametitle{Závěr}
    Naše závěry s mačkostrojem tedy jsou:
    \mypause \\ \vspace*{20pt}
    
    \Huge{\centerline{\textbf{Nedělejte to}}}
\end{frame}

\end{document}