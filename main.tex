\documentclass[]{article}

% \usepackage{czech}
\usepackage[utf8]{inputenc}   % pro unicode UTF-8

\usepackage{tabularx}

\title{FIplomová práce}
\author{Maara at. al.}

\begin{document}

%\maketitle

\section{Obálka}

\section{Zadání}

\textit{Každý ze skupiny dostane jiné zadání. Jejich témata budou odpovídat obsahu prezentace u obhajoby.}

\section{Prohlášení}

Prohlašuji, že tento víkend strávím čestně a sportovně. Moje účast na Poznej FI je zcela dobrovolná, a nepřijel jsem jenom abych dostal lepší známku z informatiky. Slibuji, že nevyliju čaj do klávesnice, že nepřejedu žádný kabel židlí, ani že nevypiju všechnu kávu v kuchyňce na čtvrtém podlaží. Dále také slibuji, že se nebudu smát Maarovi, až si z něj Domča bude utahovat.

\section{Poděkování}

Děkuji svým rodičům, že máme doma pořád plnou ledničku. Děkuji svému křečkovi Bobymu za to, že utekl zrovna před písemkou z matematiky a já ho musel nahánět, a nemusel jsem se na písemku učit. 

\section{Shrnutí}

Poznej FI! je víkendová akce pořádaná Spolkem přátel severské zvěře (dále jen zvěřinec) v prostorách Fakulty informatiky Masarykovy univerzity (FI). Studenti budou moct poznat fakultu i z jiné strany, než je běžné studium. V rámci programu je připraveno několik workshopů, na kterých se studenti seznámí s výzkumem, který na fakultě probíhá a nahlédnou do některých laboratoří, kde si budou moct tyto věci vyzkoušet. Kromě této odborné části je připraven také zážitkový program, kde studenti navštíví i ukryté kouty fakulty - a to od garáží, přes posluchárny a knihovnu až po únikový východ z budovy.

\section{Klíčová slova}

\begin{description}
    \item[Fakulta informatiky] - budova v Brně na Botanické 68a s největším střešním bazenem ve městě. Taky tam učí informatiku
    \item[Masarykova univerzita] - 
    \item[Spolek přátel severské zvěře] - studentský spolek na FI, kterému se také říká \textit{zvěřinec}.
    \item[Učebna A217] - místo, kde si během Poznej FI! můžu nechat věci
    \item[Učebna A220] - místo, kam bych neměl chodit, protože tam organizátoři mají spoustu tajných věcí, které nesmím vidět
\end{description}
     

\section{Obsah}

%\begin{table}[h!]
    \begin{tabularx}{\textwidth}{lX}
        \textbf{Sobota} &  \\ \hline
        9:00 & Začátek Poznej FI! v A217 \\
        14:00 & Budou workshopy! Můžu se těšit na \\
        & \textbf{Workshop I} o věcech, o kterých jsem předtím vůbec nevěděl. Popis by se měl vejít do tří až čtyř řádků. Pokud se nerozkecám, tak je to brnkačka! \\
        & \textbf{Workshop II} o věcech, o kterých jsem předtím vůbec nevěděl. Popis by se měl vejít do tří až čtyř řádků. Pokud se nerozkecám, tak je to brnkačka! A navíc je jiný než ten neoriginální předchozí popis \\
        22:22 & Konec dnešní části Poznej FI! Sejdeme se zase zítra v 8 \\
        & \\ 
        & \\ 
        \textbf{Neděle} & Some more text \\ \hline
        8:00 & Sraz v A217 \\
        14:00 & Poznal jsem FI! Kdy se sem můžu vrátit? \\
    \end{tabularx}
%\end{table}

\section{Úvod(ní formality)}



\appendix
\section{Závěr}

Zpětná vazba účastníků, dopis pro sebe, ...

\end{document}
