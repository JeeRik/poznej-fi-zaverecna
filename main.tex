\documentclass[]{article}

% \usepackage{czech}
\usepackage[utf8]{inputenc}   % pro unicode UTF-8

\usepackage{tabularx}
\usepackage{ifthen}

\def \studentname {Maara}
\newboolean{studentisfemale}
\setboolean{studentisfemale}{false}


\begin{document}
\pagestyle{empty}

%\maketitle

\section*{Obálka}

\pagebreak
\section*{Zadání}

\textit{Každý ze skupiny dostane jiné zadání. Jejich témata budou odpovídat částem obsahu prezentace u obhajoby.}

\pagebreak
\section*{Prohlášení}

Prohlašuji, že tento víkend strávím čestně a sportovně. Moje účast na Poznej FI je zcela dobrovolná, a nepřijel jsem jenom abych dostal lepší známku z informatiky. Slibuji, že nevyliju čaj do klávesnice, že nepřejedu žádný kabel židlí, a že nevypiju všechnu kávu v kuchyňce na čtvrtém podlaží. Dále také slibuji, že se nebudu smát Maarovi, až si z něj Domča bude utahovat.

\vspace{20pt}
\hspace*{\fill}\studentname

\pagebreak
\section*{Poděkování}

Děkuji svým rodičům, že máme doma pořád plnou ledničku. Děkuji svému křečkovi Bobymu za to, že utekl zrovna před písemkou z matematiky a já ho musel nahánět, a nemusel jsem se na písemku učit. 

\pagebreak
\section*{Shrnutí}

Poznej FI! je víkendová akce pořádaná Spolkem přátel severské zvěře (dále jen zvěřinec) v prostorách Fakulty informatiky Masarykovy univerzity (FI). Studenti budou moct poznat fakultu i z jiné strany, než je běžné studium. V rámci programu je připraveno několik workshopů, na kterých se studenti seznámí s výzkumem, který na fakultě probíhá a nahlédnou do některých laboratoří, kde si budou moct tyto věci vyzkoušet. Kromě této odborné části je připraven také zážitkový program, kde studenti navštíví i ukryté kouty fakulty - a to od garáží, přes posluchárny a knihovnu až po únikový východ z budovy.

Zvěřinec na FI dělá i jiné střeštěnosti pro středoškoláky, jako jsou InterLoS, InterSoB nebo KSI. Pro přehled těchto akcí i s jejich popisem a daty konání nahlédněte do Přílohy A.

\pagebreak
\section*{Klíčová slova}

\begin{description}
    \item[Fakulta informatiky] - budova v Brně na Botanické 68a s největším střešním bazenem ve městě. A taky tam učí informatiku a je tam zvěřinec
    \item[Masarykova univerzita] - 
    \item[Spolek přátel severské zvěře] - studentský spolek na FI, kterému se také říká \textit{zvěřinec}.
    \item[Učebna A217] - místo, kde si během Poznej FI! můžu nechat věci
    \item[Učebna A220] - místo, kam bych neměl chodit, protože tam organizátoři mají spoustu tajných věcí, které nesmím vidět
\end{description}

\pagebreak
\section*{Obsah}


\begin{tabularx}{\textwidth}{lX}
    \textbf{Sobota} &  \\ \hline
    9:00 & Začátek Poznej FI! v A217 \\
    13:00 & Oběd. Pizza z se školn menzy. \\
    14:00 & Budou workshopy! Můžu se těšit na \\
    & \textbf{Workshop I} o věcech, o kterých jsem předtím vůbec nevěděl. Popis by se měl vejít do tří až čtyř řádků. Pokud se nerozkecám, tak je to brnkačka! \\
    & \textbf{Workshop II} o jiných věcech, o kterých jsem předtím ale taky vůbec nevěděl. Popis by se prý měl vejít do tří až čtyř řádků. Tak se naschvál moc nerozepisuju, aby to vyšlo \\
    & \textbf{Workshop III} o jiných věcech, o kterých jsem předtím ale taky vůbec nevěděl. Popis by se prý měl vejít do tří až čtyř řádků. Tak se naschvál moc nerozepisuju, aby to vyšlo \\
    16:00 & Svačina. Studentský speciál - rohlíky, možná s jogurtem, pokud ještě neodešel \\
    18:00 & Večeře. Připravuje severská zvěř, takže se můžete těšit na Švédské stoly! \\
    22:22 & Konec dnešní části Poznej FI! Sejdeme se zase zítra v 8 \\
    & \\ 
    & \\ 
    \textbf{Neděle} & Some more text \\ \hline
    8:00 & Sraz v A217 \\
    & \textbf{Workshop A} kde se taky dozvím něco zajímavého. A zbytek vyplníme filler textem, aby to mělo podobný formát. \\
    & \textbf{Workshop B} a k němu text. Spousta textu. Ještě víc textu. Znáte to. Prostě hlavně aby tu něco bylo. \\
    & \textbf{Workshop C} Filler text. Filler text. Filler text. Filler text. Filler text. Filler text. Filler text. Filler text. Filler text. Filler text. \\
    14:00 & Poznal jsem FI! Kdy a jak se sem můžu vrátit? \\
\end{tabularx}


\pagebreak
\section*{Vítejte na Poznej FI!}

Co budete potřebovat, kdy a kde.

Informace k jídlu

Informace k ubytování



\pagebreak
\section*{Závěr}

Zpětná vazba účastníků, dopis pro sebe, ...


\pagebreak
\section*{Literatura}

\pagebreak
\appendix
\section*{Další akce pořádané zvěřincem}

\pagebreak
\section*{Materiály k workshopům}

Doplňující materiály k workshopům, které se nevejdou přímo do vlastní práce.

\end{document}
